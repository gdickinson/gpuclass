\documentclass{article}
\usepackage{fullpage}
\usepackage{mathpazo}
\usepackage{hyperref}

\begin{document}
  
  \title{GPU Project Report 2: Accelerated RC4 Stream Cipher Detailed Design}
  \author{Guy Dickinson (guy.dickinson@nyu.edu) \& William Ward, (wwward@nyu.edu)}
  \maketitle
  
  \section{Previous Work: Survey}
  
  \section{Previous Work: Shortcomings}
   Previous work in this area has focused on encrypting and decrypting static data in fixed files, in particular in PDFs \footnote{\url{http://ieeexplore.ieee.org/xpls/abs_all.jsp?arnumber=5276924&tag=1}}. We propose to examine encrypting and decrypting data of arbitrary length, such as one might find in a network rather than a filesystem environment. Such an environment precludes generating the entire necessary keystream start time and instead requires a mechanism to generate it on-the-fly, move it efficiently to the GPU along with the target text, and recovering the results. Keeping sufficient keying material on hand at all times is a particular challenge. Fortunately, unlike many block ciphers which depend on previous ciphertext output to generate more keying material (a process known as cipher block chaining), the keystream is generated completely independently of either plaintext or ciphertext.
   
   Additionally, it may be possible to generate the keystream on-device rather than on the CPU (even though it is a fundamentally serial operation) -- this would substantially reduce the amount of memory traffic between host and device.
   

  \section{Data Structure}
  
  \begin{enumerate}
    \item A simple array of bytes will store the keystream, plaintext, and ciphertext.  The size of the array will be determined by the amount of material that must be buffered from the input source, and the resulting processed output.  The depth of this buffer and performance ratio may vary depending on the achievable stream rate, and the overhead cost of moving the buffer in and out of the device.
    \item A 256 byte array, called ``S," which is permuted to generate the keystream.
    \item Two index pointers, ``i" and ``j," which are used by the PRGA to facilitate swapping elements within array ``S".
  \end{enumerate}

  A more complete discussion of the data structures involved may be found via any public source detailing RC4, including Wikipedia.
  
  \section{Main Functions: CPU}
  
  \begin{enumerate}
    \item Key-scheduling algorithm - this function initializes the array with the encryption key and is used by the PRGA to produce the keystream.
    \item Pseudo-random generation algorithm - this function produces the keystream in a sequential manner from the array derived by the key scheduler.
    \item Buffer sizing and external I/O - functions required to determine array size, transfer frequency, and input-output.
  \end{enumerate}


  \section{Main Functions: Device}
 Data enciphering - this function takes plaintext or ciphertext and XORs its contents with the keystream, this operation is suitable for parallel execution.

  \section{Optimizations}
  \begin{enumerate}
    \item Keystream generation - finding methods to benefit from parallelization or reasonably fast keystream generation on-device.
    \item Efficient memory transfers - reducing the overhead of shipping data between CPU and device by coalescing transfers
  \end{enumerate}

  
\end{document}