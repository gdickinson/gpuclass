\documentclass{article}
\usepackage{fullpage}
\usepackage{mathpazo}
\usepackage{hyperref}

\begin{document}
  
  \title{GPU Project Definition: Accelerated RC4 Stream Cipher}
  \author{Guy Dickinson (guy.dickinson@nyu.edu) \& William Ward, (wwward@nyu.edu)}
  \maketitle
  
  \section{About}
  RC4 is a symmetric stream cipher originally developed in 1987 and released/discovered in 1994. \footnote{\url{http://www.wisdom.weizmann.ac.il/~itsik/RC4/rc4.html}}
  It underpins a wide range of cryptographic applications, including SSL/TLS, WEP (now deprecated), and its successor, WPA. In contrast with block ciphers, symmetric ciphers generate keying material based on a static initialization vector which is then XOR'ed with the plaintext input to generate ciphertext which can be recovered simply by applying the same XOR operation with the same keying material. This gives rise to a certain amount of parallelism; all the XOR operations may be performed independent of each other given sufficient keying material on hand. We aim to accelerate both encryption and decryption using RC4 by parallelizing the exclusive-or operation, and explore the feasibility of parallelizing keystream generation within the pseudo-random generation algorithm (PRGA) which must be run as many time as necessary to generate keystream equal or greater in length to the plaintext to be enciphered.
  There several possible pitfalls with regard to on-device memory management and limiting the round-trips to system memory which must be taken into account, particularly with regard to those applications which require relatively low latency throughput such as network performance. We expect this project to be a good training ground for learning about the nuances of CUDA memory management.
  
  \section{Input/Output}
  We expect to be able to take plaintext of arbitrary length and reliably encipher it using RC4, producing the precise same result as a known-good RC4 implementation as distributed in a well-used package such as OpenSSL or GPG. Similarly, we expect to be able to take ciphertext generated by an arbitrary implementation of RC4 (including our own) and recover the plaintext given the private key.
  
  \section{Options for Further Research}
  Further analysis of parallelization may involve examining the impact of parallel RC4 cipher processing upon applications further up the stack. For example, we could test the speedup RC4/WPA encrypted networks, or RC4-TLS encrypted HTTP connections relative to a serial implementation of the same.
  
  
\end{document}